In this section, we describe the experimental setup.
Our system can be monitored via descriptive messages which are sent to the standard output of the running programs.

\subsection{DAS-4}
Our cloud path tracer system has been set up on the DAS-4\cite{das4}.
This is a distributed supercomputer which consists of six different clusters, provided by several Dutch universities and institutions.

Using OpenNebula, provided by DAS-4, 11 different virtual machines are created.
Ten of them for worker nodes, and another with an extra 8GB of harddisk space for the master node.

Workers will always be running their program, but essentially be idle when not assigned any work.
This is because of the blocking socket receive call, which causes the program to halt until it receives a task from the master node.
This does mean, however, that the starting and stopping of virtual machines is not taken into account for any of our experiments.

\subsection{Multitenancy and elasticity experiment}
In this experiment, five jobs are gradually sent to the master node by two users.
This happens in such a way that multiple jobs will be running at a time, up to a point where we can no longer assign the maximal amount of workers per job.

This will show how multiple users can simultaneously use the system to perform jobs for them.
Also, it shows how the amount of working workers scales up with demand and back down again.

\subsection{Load balancing experiment}
For this experiment, ten jobs will be run with $mwpj = 3$ and three workers.
The total processing time that each worker does will be measured.

This will demonstrate how well balanced the load to workers turns out to be.

\subsection{Reliability experiment}
In this experiment, ten jobs will be run with $mwpj = 3$ and ten workers.
A worker process is manually killed and restarted at some later point in time.
This is repeated for 25 workers.

This will show that the system deals with failing worker nodes.