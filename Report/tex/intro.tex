Path tracing is a rendering technique that simulates the paths that photons take.
The main advantage of this technique, compared to the usual rasterization approach, is that it correctly calculates global illumination.
Instead of only taking direct lighting into account and faking global illumination by an ambient term, path tracing also takes indirect lighting into account.
This is where light hits a surface, bounces off and illuminates another surface.
Another advantage is that there is no need to approximate lights as point lights; light can easily be emitted from surfaces.
Other effects such as antialiasing and depth of field are easily implemented with this rendering technique as well.

The movie industry uses a lot of computer generated imagery to enhance the visuals of their movies.
At animation studios, even their whole movie is rendered by computers.
Most of these movies use some form of a path tracer to produce this imagery, since the results converge to a solution of the rendering equation\cite{kajiya}.
This means that the calculated lighting closely approximates reality.

Since the calculation for the light along every ray does not depend on another ray, they can be calculated in parallel.
In our previous work, called CUDAPT\cite{cudapt}, we implemented a path tracer that parallellizes per pixel using the power of modern graphics cards.
Using CUDA, each small graphics card processor would calculate the color of a few pixels by tracing a set number of sample light paths through the pixels.

For this paper, the parallellization happens in a slightly different way.
Instead of using the many small processors on the graphics card for parallellization, virtual machines in a cloud are used, which we will refer to as nodes from now on.
Every node now calculates a frame of a movie, instead of just a few pixels of such a frame.
These nodes are called worker nodes.
This was done to make sure that communication between nodes would not be the bottleneck in this system.

In this system, there is one master node.
This node receives jobs, which are descriptions of how to render a movie, and makes sure that the worker nodes work together to finish these jobs.

In Section~\ref{sec:systemoverview}, an overview of the system is provided, which will not go into detail about the workings of a path tracer.
Section~\ref{sec:implementationdetails} gives any further explanation required for the implementation of the system.
An overview of the experiments done can be found in Section~\ref{sec:experimentalsetup}.
In Section~\ref{sec:results}, the results of these experiments can be found.
Finally, in Section~\ref{sec:conclusion}, we make some concluding remarks on our work.
